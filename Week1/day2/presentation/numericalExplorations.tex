\documentclass{beamer}\usepackage[]{graphicx}\usepackage[]{color}
%% maxwidth is the original width if it is less than linewidth
%% otherwise use linewidth (to make sure the graphics do not exceed the margin)
\makeatletter
\def\maxwidth{ %
  \ifdim\Gin@nat@width>\linewidth
    \linewidth
  \else
    \Gin@nat@width
  \fi
}
\makeatother

\definecolor{fgcolor}{rgb}{0.345, 0.345, 0.345}
\newcommand{\hlnum}[1]{\textcolor[rgb]{0.686,0.059,0.569}{#1}}%
\newcommand{\hlstr}[1]{\textcolor[rgb]{0.192,0.494,0.8}{#1}}%
\newcommand{\hlcom}[1]{\textcolor[rgb]{0.678,0.584,0.686}{\textit{#1}}}%
\newcommand{\hlopt}[1]{\textcolor[rgb]{0,0,0}{#1}}%
\newcommand{\hlstd}[1]{\textcolor[rgb]{0.345,0.345,0.345}{#1}}%
\newcommand{\hlkwa}[1]{\textcolor[rgb]{0.161,0.373,0.58}{\textbf{#1}}}%
\newcommand{\hlkwb}[1]{\textcolor[rgb]{0.69,0.353,0.396}{#1}}%
\newcommand{\hlkwc}[1]{\textcolor[rgb]{0.333,0.667,0.333}{#1}}%
\newcommand{\hlkwd}[1]{\textcolor[rgb]{0.737,0.353,0.396}{\textbf{#1}}}%

\usepackage{framed}
\makeatletter
\newenvironment{kframe}{%
 \def\at@end@of@kframe{}%
 \ifinner\ifhmode%
  \def\at@end@of@kframe{\end{minipage}}%
  \begin{minipage}{\columnwidth}%
 \fi\fi%
 \def\FrameCommand##1{\hskip\@totalleftmargin \hskip-\fboxsep
 \colorbox{shadecolor}{##1}\hskip-\fboxsep
     % There is no \\@totalrightmargin, so:
     \hskip-\linewidth \hskip-\@totalleftmargin \hskip\columnwidth}%
 \MakeFramed {\advance\hsize-\width
   \@totalleftmargin\z@ \linewidth\hsize
   \@setminipage}}%
 {\par\unskip\endMakeFramed%
 \at@end@of@kframe}
\makeatother

\definecolor{shadecolor}{rgb}{.97, .97, .97}
\definecolor{messagecolor}{rgb}{0, 0, 0}
\definecolor{warningcolor}{rgb}{1, 0, 1}
\definecolor{errorcolor}{rgb}{1, 0, 0}
\newenvironment{knitrout}{}{} % an empty environment to be redefined in TeX

\usepackage{alltt}
\mode<presentation>{}
\usetheme{CambridgeUS}
\usecolortheme{sidebartab}
%\pdfmapfile{+sansmathaccent.map}
\title{Numerical explorations with R}
% \subtitle{means, median, SD; use R as calc}
\author{Amos Thairu}
\IfFileExists{upquote.sty}{\usepackage{upquote}}{}
\begin{document}
% \SweaveOpts{concordance=TRUE}
\begin{frame}
\titlepage
\end{frame}

% data <- read.csv('H:/PwaniStatsCourse/PwaniR_Training_2015/PwaniR_Training_2015/Week1/day2/presentation/bwmal.csv')
% bwmal_subset <- data[c(1:10,190:200),-c(1,9:12)]
% row.names(data) <- NULL
% write.csv(bwmal_subset, 'H:/PwaniStatsCourse/PwaniR_Training_2015/PwaniR_Training_2015/Week1/day2/presentation/bwmal_subset.csv')

\begin{frame}[fragile]\frametitle{Reading data files}
Import dataset
\begin{knitrout}
\definecolor{shadecolor}{rgb}{0.969, 0.969, 0.969}\color{fgcolor}\begin{kframe}
\begin{alltt}
\hlstd{data} \hlkwb{<-} \hlkwd{read.csv}\hlstd{(}\hlstr{"bwmal_subset.csv"}\hlstd{)}
\end{alltt}
\end{kframe}
\end{knitrout}
\pause
\begin{itemize}
  \item Once read in, assigning the loaded data to objects
  \item Datasets in R are typically stored as data frames, which have a matrix structure 
  \item Observations are arranged as rows and variables, either numerical or categorical, are arranged as columns
  \item Dataset contains 8 demographic variables for 20 individuals
\end{itemize}
\pause
Get the dimension of the dataset
\begin{knitrout}
\definecolor{shadecolor}{rgb}{0.969, 0.969, 0.969}\color{fgcolor}\begin{kframe}
\begin{alltt}
\hlkwd{dim}\hlstd{(data)}
\end{alltt}
\begin{verbatim}
## [1] 21  8
\end{verbatim}
\end{kframe}
\end{knitrout}
\end{frame}

\begin{frame}[fragile]\frametitle{Viewing data}
\small
Explore variable names of the dataset
\begin{knitrout}
\definecolor{shadecolor}{rgb}{0.969, 0.969, 0.969}\color{fgcolor}\begin{kframe}
\begin{alltt}
\hlkwd{names}\hlstd{(data)}
\end{alltt}
\begin{verbatim}
## [1] "X"        "matage"   "mheight"  "gestwks"  "sex"      "bweight" 
## [7] "smoke"    "pfplacen"
\end{verbatim}
\end{kframe}
\end{knitrout}
R has ways to look at the dataset at a glance
\begin{knitrout}
\definecolor{shadecolor}{rgb}{0.969, 0.969, 0.969}\color{fgcolor}\begin{kframe}
\begin{alltt}
\hlkwd{head}\hlstd{(data)}  \hlcom{#Returns first six rows of dataset}
\end{alltt}
\begin{verbatim}
##   X matage mheight gestwks sex bweight smoke pfplacen
## 1 1     26   1.575      40   0    3.11     0        0
## 2 2     23   1.529      40   0    2.65     0        0
## 3 3     18   1.540      40   1    3.41     0        0
## 4 4     25   1.581      40   1    2.99     0        0
## 5 5     25   1.555      40   1    3.16     0        0
## 6 6     21   1.561      40   1    2.82     0        0
\end{verbatim}
\end{kframe}
\end{knitrout}
\end{frame}

\begin{frame}[fragile]\frametitle{Viewing data}
\small
We can access variables directly by using their names, using the object \$ variable notation
\begin{knitrout}
\definecolor{shadecolor}{rgb}{0.969, 0.969, 0.969}\color{fgcolor}\begin{kframe}
\begin{alltt}
\hlstd{data}\hlopt{$}\hlstd{gestwks}
\end{alltt}
\begin{verbatim}
##  [1] 40 40 40 40 40 40 41 38 40 41 39 38 39 39 39 37 39 39 39 40 39
\end{verbatim}
\end{kframe}
\end{knitrout}
Check last six rows of the dataset
\begin{knitrout}
\definecolor{shadecolor}{rgb}{0.969, 0.969, 0.969}\color{fgcolor}\begin{kframe}
\begin{alltt}
\hlkwd{tail}\hlstd{(data)}  \hlcom{#Returns first six rows of dataset}
\end{alltt}
\begin{verbatim}
##      X matage mheight gestwks sex bweight smoke pfplacen
## 16 195     19   1.583      37   0    2.42     0        0
## 17 196     20   1.534      39   1    2.93     0        0
## 18 197     30   1.543      39   0    2.59     1        1
## 19 198     38   1.602      39   0    2.48     1        0
## 20 199     20   1.540      40   1    3.02     1        1
## 21 200     24   1.503      39   0    2.79     0        1
\end{verbatim}
\end{kframe}
\end{knitrout}
\end{frame}
% 
\begin{frame}[fragile]\frametitle{Viewing data}
% \small
To access a certain entry, we most commonly use object[row,column] \\
Single cell value
\begin{knitrout}
\definecolor{shadecolor}{rgb}{0.969, 0.969, 0.969}\color{fgcolor}\begin{kframe}
\begin{alltt}
\hlstd{data[}\hlnum{2}\hlstd{,} \hlnum{3}\hlstd{]}
\end{alltt}
\begin{verbatim}
## [1] 1.529
\end{verbatim}
\end{kframe}
\end{knitrout}
Omitting row value implies all rows; here all rows in column 3
\begin{knitrout}
\definecolor{shadecolor}{rgb}{0.969, 0.969, 0.969}\color{fgcolor}\begin{kframe}
\begin{alltt}
\hlstd{data[,} \hlnum{3}\hlstd{]}
\end{alltt}
\begin{verbatim}
##  [1] 1.575 1.529 1.540 1.581 1.555 1.561 1.590 1.502 1.666 1.591 1.567
## [12] 1.566 1.540 1.502 1.560 1.583 1.534 1.543 1.602 1.540 1.503
\end{verbatim}
\end{kframe}
\end{knitrout}
\end{frame}

\begin{frame}[fragile]\frametitle{More data viewing}
Omitting column values implies all columns; here all columns in row 2
\begin{knitrout}
\definecolor{shadecolor}{rgb}{0.969, 0.969, 0.969}\color{fgcolor}\begin{kframe}
\begin{alltt}
\hlstd{data[}\hlnum{2}\hlstd{, ]}
\end{alltt}
\begin{verbatim}
##   X matage mheight gestwks sex bweight smoke pfplacen
## 2 2     23   1.529      40   0    2.65     0        0
\end{verbatim}
\end{kframe}
\end{knitrout}
Can also use ranges - rows 2 and 3, columns 2 and 3
\begin{knitrout}
\definecolor{shadecolor}{rgb}{0.969, 0.969, 0.969}\color{fgcolor}\begin{kframe}
\begin{alltt}
\hlstd{data[}\hlnum{2}\hlopt{:}\hlnum{3}\hlstd{,} \hlnum{2}\hlopt{:}\hlnum{3}\hlstd{]}
\end{alltt}
\begin{verbatim}
##   matage mheight
## 2     23   1.529
## 3     18   1.540
\end{verbatim}
\end{kframe}
\end{knitrout}
\end{frame}

% \normal
\begin{frame}[fragile]\frametitle{Data summaries}
\begin{itemize}
  \item Enables us to see the main characteristics of data before any formal modeling or hypothesis testing
  \item Particular techniques depends on the type of variable: Continuous or categorical
  \item Continuous eg. age, height
  \item Categorical eg. smoking status, sex
\end{itemize}
\end{frame}

\begin{frame}[fragile]\frametitle{Some data explorations: Continuous variables}
\begin{knitrout}
\definecolor{shadecolor}{rgb}{0.969, 0.969, 0.969}\color{fgcolor}\begin{kframe}
\begin{alltt}
\hlcom{# some data explorations}
\hlkwd{mean}\hlstd{(data}\hlopt{$}\hlstd{mheight)}
\end{alltt}
\begin{verbatim}
## [1] 1.559
\end{verbatim}
\end{kframe}
\end{knitrout}
\pause
\begin{knitrout}
\definecolor{shadecolor}{rgb}{0.969, 0.969, 0.969}\color{fgcolor}\begin{kframe}
\begin{alltt}
\hlkwd{var}\hlstd{(data}\hlopt{$}\hlstd{mheight)}
\end{alltt}
\begin{verbatim}
## [1] 0.001461
\end{verbatim}
\end{kframe}
\end{knitrout}
\pause
\begin{knitrout}
\definecolor{shadecolor}{rgb}{0.969, 0.969, 0.969}\color{fgcolor}\begin{kframe}
\begin{alltt}
\hlkwd{sd}\hlstd{(data}\hlopt{$}\hlstd{matage)}
\end{alltt}
\begin{verbatim}
## [1] 5.528
\end{verbatim}
\begin{alltt}
\hlkwd{median}\hlstd{(data}\hlopt{$}\hlstd{matage)}
\end{alltt}
\begin{verbatim}
## [1] 22
\end{verbatim}
\end{kframe}
\end{knitrout}
% \pause
% <<tidy=TRUE>>=
% median(data$matage)
% @
\end{frame}

\begin{frame}[fragile]\frametitle{More data explorations}
Produce various summaries of continous variable
\begin{knitrout}
\definecolor{shadecolor}{rgb}{0.969, 0.969, 0.969}\color{fgcolor}\begin{kframe}
\begin{alltt}
\hlkwd{summary}\hlstd{(data}\hlopt{$}\hlstd{matage)}  \hlcom{#sumarize continous variable}
\end{alltt}
\begin{verbatim}
##    Min. 1st Qu.  Median    Mean 3rd Qu.    Max. 
##    17.0    20.0    22.0    23.6    25.0    38.0
\end{verbatim}
\end{kframe}
\end{knitrout}
\end{frame}

\begin{frame}[fragile]\frametitle{Explorations for categorical variables}
Summarize single categorical variable
\begin{knitrout}
\definecolor{shadecolor}{rgb}{0.969, 0.969, 0.969}\color{fgcolor}\begin{kframe}
\begin{alltt}
\hlkwd{table}\hlstd{(data}\hlopt{$}\hlstd{sex)}
\end{alltt}
\begin{verbatim}
## 
##  0  1 
## 11 10
\end{verbatim}
\end{kframe}
\end{knitrout}
Cross-tabulation of two categorical variables
\begin{knitrout}
\definecolor{shadecolor}{rgb}{0.969, 0.969, 0.969}\color{fgcolor}\begin{kframe}
\begin{alltt}
\hlkwd{table}\hlstd{(data}\hlopt{$}\hlstd{sex, data}\hlopt{$}\hlstd{smoke)}
\end{alltt}
\begin{verbatim}
##    
##     0 1
##   0 8 3
##   1 9 1
\end{verbatim}
\end{kframe}
\end{knitrout}
\end{frame}

\begin{frame}[fragile]\frametitle{Alternative cross tabulation}
Using \emph{'with'} command includes variable labels in the table
\begin{knitrout}
\definecolor{shadecolor}{rgb}{0.969, 0.969, 0.969}\color{fgcolor}\begin{kframe}
\begin{alltt}
\hlkwd{with}\hlstd{(data,} \hlkwd{table}\hlstd{(sex, smoke))}
\end{alltt}
\begin{verbatim}
##    smoke
## sex 0 1
##   0 8 3
##   1 9 1
\end{verbatim}
\end{kframe}
\end{knitrout}
\end{frame}

\begin{frame}[fragile]\frametitle{Use R as calculator}
\begin{knitrout}
\definecolor{shadecolor}{rgb}{0.969, 0.969, 0.969}\color{fgcolor}\begin{kframe}
\begin{alltt}
\hlnum{1000} \hlopt{-} \hlnum{2} \hlopt{*} \hlnum{10}\hlopt{^}\hlnum{2}\hlopt{/}\hlstd{(}\hlnum{8} \hlopt{+} \hlnum{2}\hlstd{)}  \hlcom{#expression to evaluate}
\end{alltt}
\begin{verbatim}
## [1] 980
\end{verbatim}
\begin{alltt}
\hlcom{# Built-in functions:}
\hlkwd{log}\hlstd{(}\hlnum{1.4}\hlstd{)}  \hlcom{#returns the natural logarithm of the number 1.4}
\end{alltt}
\begin{verbatim}
## [1] 0.3365
\end{verbatim}
\begin{alltt}
\hlkwd{log10}\hlstd{(}\hlnum{1.4}\hlstd{)}  \hlcom{# returns the log to the base of 10}
\end{alltt}
\begin{verbatim}
## [1] 0.1461
\end{verbatim}
\begin{alltt}
\hlkwd{sqrt}\hlstd{(}\hlnum{16}\hlstd{)}  \hlcom{#returns the square root of 16}
\end{alltt}
\begin{verbatim}
## [1] 4
\end{verbatim}
\end{kframe}
\end{knitrout}
\end{frame}

\begin{frame}[fragile]\frametitle{Calculations with assignment statements}
We can store a value(s) in an R object using the assignment symbol $<-$ ("less than" followed by a hyphen)
\begin{knitrout}
\definecolor{shadecolor}{rgb}{0.969, 0.969, 0.969}\color{fgcolor}\begin{kframe}
\begin{alltt}
\hlstd{x} \hlkwb{<-} \hlnum{2.5}
\end{alltt}
\end{kframe}
\end{knitrout}
To check what is in a variable type the variable name
\begin{knitrout}
\definecolor{shadecolor}{rgb}{0.969, 0.969, 0.969}\color{fgcolor}\begin{kframe}
\begin{alltt}
\hlstd{x}
\end{alltt}
\begin{verbatim}
## [1] 2.5
\end{verbatim}
\end{kframe}
\end{knitrout}
Can store a computation under a new R object or change the current value stored in an old object
\begin{knitrout}
\definecolor{shadecolor}{rgb}{0.969, 0.969, 0.969}\color{fgcolor}\begin{kframe}
\begin{alltt}
\hlstd{y} \hlkwb{<-} \hlnum{3} \hlopt{*} \hlkwd{log}\hlstd{(x)}
\end{alltt}
\end{kframe}
\end{knitrout}
\end{frame}

\end{document}
