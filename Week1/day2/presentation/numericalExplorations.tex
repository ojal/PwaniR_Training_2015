\documentclass{beamer}
\mode<presentation>{}
\usetheme{CambridgeUS}
\usecolortheme{sidebartab}
%\pdfmapfile{+sansmathaccent.map}
\title{Numerical explorations with R}
\subtitle{means, median, SD; use R as calc}
\author{Amos Thairu}
\usepackage{Sweave}
\begin{document}
\Sconcordance{concordance:numericalExplorations.tex:numericalExplorations.Rnw:%
1 13 1 50 0 5 1 3 0 4 1 7 0 4 1 8 0 6 1 20 0 7 1 7 0 4 1 7 0 4 1 7 %
0 4 1 7 0 6 1 8 0 6 1 9 0 4 1 10 0 4 1 10 0 7 1 7 0 6 1 17 0 9 1}

\begin{frame}
\titlepage
\end{frame}

\begin{frame}[fragile]\frametitle{Import data}
\begin{Schunk}
\begin{Sinput}
> #import dataset
> data <- read.csv("bwmal.csv")
\end{Sinput}
\end{Schunk}
\pause
\begin{Schunk}
\begin{Sinput}
> dim(data) #Returns the dimension of the dataset
\end{Sinput}
\begin{Soutput}
[1] 791  12
\end{Soutput}
\end{Schunk}
\pause
\begin{Schunk}
\begin{Sinput}
> names(data) #Returns variable names of the dataset
\end{Sinput}
\begin{Soutput}
 [1] "X"        "matage"   "mheight"  "gestwks"  "sex"      "bweight" 
 [7] "smoke"    "pfplacen" "parity"   "workload" "matagegp" "gestcat" 
\end{Soutput}
\end{Schunk}
\end{frame}

\begin{frame}[fragile]\frametitle{View data}
\begin{Schunk}
\begin{Sinput}
> head(data) #Returns first six rows of dataset
\end{Sinput}
\begin{Soutput}
  X matage mheight gestwks sex bweight smoke pfplacen parity workload matagegp
1 1     26   1.575      40   0    3.11     0        0      3        0        3
2 2     23   1.529      40   0    2.65     0        0      1        0        3
3 3     18   1.540      40   1    3.41     0        0      0        0        1
4 4     25   1.581      40   1    2.99     0        0      2        0        3
5 5     25   1.555      40   1    3.16     0        0      1        1        3
6 6     21   1.561      40   1    2.82     0        0      1        1        2
  gestcat
1       2
2       2
3       2
4       2
5       2
6       2
\end{Soutput}
\end{Schunk}
\end{frame}

\begin{frame}[fragile]\frametitle{Some data explorations}
\begin{Schunk}
\begin{Sinput}
> #some data explorations
> mean(data$mheight)
\end{Sinput}
\begin{Soutput}
[1] 1.543273
\end{Soutput}
\end{Schunk}
\pause
\begin{Schunk}
\begin{Sinput}
> var(data$mheight)
\end{Sinput}
\begin{Soutput}
[1] 0.002884892
\end{Soutput}
\end{Schunk}
\pause
\begin{Schunk}
\begin{Sinput}
> sd(data$matage)
\end{Sinput}
\begin{Soutput}
[1] 5.139645
\end{Soutput}
\end{Schunk}
\pause
\begin{Schunk}
\begin{Sinput}
> median(data$matage)
\end{Sinput}
\begin{Soutput}
[1] 23
\end{Soutput}
\end{Schunk}
\end{frame}

\begin{frame}[fragile]\frametitle{More data explorations}
\begin{Schunk}
\begin{Sinput}
> summary(data$matage)  #sumarize continous variable
\end{Sinput}
\begin{Soutput}
   Min. 1st Qu.  Median    Mean 3rd Qu.    Max. 
  13.00   20.00   23.00   23.78   27.00   46.00 
\end{Soutput}
\end{Schunk}
\end{frame}

\begin{frame}[fragile]\frametitle{Explorations for categorical variables}
\begin{Schunk}
\begin{Sinput}
> table(data$sex)  #summarize categorical variable
\end{Sinput}
\begin{Soutput}
  0   1 
381 410 
\end{Soutput}
\end{Schunk}
\pause
\begin{Schunk}
\begin{Sinput}
> table(data$sex,data$smoke) #cross-tabulation of two categorical variables
\end{Sinput}
\begin{Soutput}
      0   1
  0 346  35
  1 378  32
\end{Soutput}
\end{Schunk}
\pause
\begin{Schunk}
\begin{Sinput}
> with(data, table(sex,smoke)) #Alternatively, with variable labels
\end{Sinput}
\begin{Soutput}
   smoke
sex   0   1
  0 346  35
  1 378  32
\end{Soutput}
\end{Schunk}
\end{frame}

\begin{frame}[fragile]\frametitle{Use R as calculator}
\begin{Schunk}
\begin{Sinput}
> #enter the expression that we want evaluated and hit enter
> 1000-2*10^2/(8+2)
\end{Sinput}
\begin{Soutput}
[1] 980
\end{Soutput}
\end{Schunk}
\begin{Schunk}
\begin{Sinput}
> #Built-in functions:
> log(1.4)  #returns the natural logarithm of the number 1.4
\end{Sinput}
\begin{Soutput}
[1] 0.3364722
\end{Soutput}
\begin{Sinput}
> log10(1.4)  # returns the log to the base of 10
\end{Sinput}
\begin{Soutput}
[1] 0.146128
\end{Soutput}
\begin{Sinput}
> sqrt(16)  #returns the square root of 16
\end{Sinput}
\begin{Soutput}
[1] 4
\end{Soutput}
\end{Schunk}
\end{frame}

\begin{frame}[fragile]\frametitle{Calculations with assignment statements}
\begin{Schunk}
\begin{Sinput}
> #we can store a value(s) under some variable name using the assignment symbol <- 
> #("less than" followed by a hyphen)
> x <- 2.5
> #to know what is in a variable type the variable name
> x
\end{Sinput}
\begin{Soutput}
[1] 2.5
\end{Soutput}
\begin{Sinput}
> #can store a computation under a new variable name or change the current value in an old variable
> y <- 3*log(x)
\end{Sinput}
\end{Schunk}
\end{frame}

\end{document}

%more on using R as a calculator
%more on summary stats
%max(x) 
%?with


