\documentclass[slidestop,compress,14pt,xcolor=dvipsnames]{beamer}
\usepackage{lmodern}
\usepackage{graphicx} %package for attaching images
%\usetheme{Madrid}
%\usetheme{Ilmenau}
\usetheme{CambridgeUS}
 %verbatim
\mode<presentation>
\setbeamercolor{section in head}{parent=palette quaternary}

\makeatletter
\setbeamertemplate{}
{%
\vskip-9ex%
\begin{beamercolorbox}{}
\hfill\usebeamercolor[fg]{navigation symbols dimmed}%
    \insertslidenavigationsymbol
    \insertframenavigationsymbol
    \insertsubsectionnavigationsymbol
    \insertsectionnavigationsymbol
    \insertdocnavigationsymbol
    \insertbackfindforwardnavigationsymbol
  \end{beamercolorbox}%    
  \begin{beamercolorbox}[ht=2ex,dp=3ex]{section in head}%
    \insertnavigation{\paperwidth}
  \end{beamercolorbox}%
}%
\makeatother

\author{Ken Mwai}
\institute{Pwani-University R WorkShop}

\usepackage{Sweave}
\begin{document}
\Sconcordance{concordance:day4.Lesson2.correlation.tex:day4.Lesson2.correlation.Rnw:%
1 32 1 1 0 73 1 1 7 1 2 19 0 1 2 1 11 9 1 1 2 18 0 1 1 5 0 1 1 5 0 1 1 %
6 0 1 2 7 1 1 2 6 0 1 1 15 0 1 2 36 1}

\section{Correlation}
\begin{frame}{Title}
\vspace*{\fill}
\begin{center}
Correlation
\end{center}
\vspace*{\fill}
\end{frame}

\section{Overview}
\subsection{Introduction}
\begin{frame}{Overview}
\begin{itemize}
    \item  Correlation is made of {\bf Co- } (meaning "together"), and {\bf Relation }
    \item Statistical procedure used to measure and describe the relationship between two variables 
    \item Range between +1 and -1
      \begin{itemize}
        \item Positive when the values increase together
        \item Negative when one value decreases as the other increases 
      \end{itemize}
\ldots
\end{itemize}
\end{frame}
\begin{frame}{Overview cont..}
\begin{itemize}
  \item +1 is a perfect positive correlation
  \item 0 is no correlation (independence)
  \item -1 is a perfect negative correlation
\end{itemize}
\includegraphics[scale=0.5]{corelation}
\end{frame}
\subsection{Uses}
\begin{frame}{Use of Corelation}
When two variables, let's call them X  Y, are correlated, then one variable can be
used to predict the other variable \newline
Example:IQ and perfomance...

\end{frame}

\section{Types}
\begin{frame}{Types}
\begin{itemize}
  \item {\bf Pearson product-moment correlation} -When both variables, X and Y, are continuous
  \item {\bf Point bi-serial correlation} - When 1 variable is continuous and 1 is dichotomous
  \item {\bf Phi coefficient} - When both variables are dichotomous
  \item {\bf Spearman rank correlation} - When both variables are ordinal (ranked data)
\end{itemize}
\end{frame}


\section{Calculation}
\begin{frame}{Calculation of Correlation}
defined as \newline 
\begin{center}
$r = S_{xy}/\sqrt{S_{xx}S_{yy}}.$ 
\end{center}
where 
\begin{center} $S_{xx} = \sum\limits_{i = 1}^N {\left( {x_i - \bar x} \right)^2}$ {(variance of x)} \end{center}
and
\begin{center} 
$S_{xy} = \sum\limits_{i = 1}^N {\left( {x_i - \bar x} \right)} {\left( {y_i - \bar y} \right)}$ {(covariance of x and y)}
\end{center}
\end{frame}


\section{Exercise}
\subsection{manual}
%\begin{frame}{Exercise - Desk Work}

%\begin{tabular}
\begingroup
\fontsize{7pt}{9pt}\selectfont

\begin{Schunk}
\begin{Sinput}
> print(df)
\end{Sinput}
\begin{Soutput}
   temp icecream
1  14.2      215
2  16.4      325
3  11.9      185
4  15.2      332
5  18.5      406
6  22.1      522
7  19.4      412
8  25.1      614
9  23.4      544
10 18.1      421
11 22.6      445
12 17.2      408
\end{Soutput}
\end{Schunk}



%\end{tabular}
%\end{frame}

\subsection{Output}
%\begin{frame}{Another title}

\clearpage 

\begin{Schunk}
\begin{Sinput}
> print(df)
\end{Sinput}
\begin{Soutput}
   temp icecream deviationTemp deviationIce         SSxy      SSxx        SSyy
1  14.2      215        -4.475  -187.416667  838.6895833 20.025625 35125.00694
2  16.4      325        -2.275   -77.416667  176.1229167  5.175625  5993.34028
3  11.9      185        -6.775  -217.416667 1472.9979167 45.900625 47270.00694
4  15.2      332        -3.475   -70.416667  244.6979167 12.075625  4958.50694
5  18.5      406        -0.175     3.583333   -0.6270833  0.030625    12.84028
6  22.1      522         3.425   119.583333  409.5729167 11.730625 14300.17361
7  19.4      412         0.725     9.583333    6.9479167  0.525625    91.84028
8  25.1      614         6.425   211.583333 1359.4229167 41.280625 44767.50694
9  23.4      544         4.725   141.583333  668.9812500 22.325625 20045.84028
10 18.1      421        -0.575    18.583333  -10.6854167  0.330625   345.34028
11 22.6      445         3.925    42.583333  167.1395833 15.405625  1813.34028
12 17.2      408        -1.475     5.583333   -8.2354167  2.175625    31.17361
\end{Soutput}
\begin{Sinput}
> print(sum.SSxy)
\end{Sinput}
\begin{Soutput}
[1] 5325.025
\end{Soutput}
\begin{Sinput}
> print(sum.SSxx)
\end{Sinput}
\begin{Soutput}
[1] 176.9825
\end{Soutput}
\begin{Sinput}
> print(sum.SSyy)
\end{Sinput}
\begin{Soutput}
[1] 174754.9
\end{Soutput}
\end{Schunk}



%\end{frame}
\section{Corelation}
\subsection{Corelation in R}
\clearpage
%\begin{frame}{Corelation in R}
\begin{Schunk}
\begin{Sinput}
> cor(df$temp,df$icecream)
\end{Sinput}
\begin{Soutput}
[1] 0.9575066
\end{Soutput}
\begin{Sinput}
> cor.test(df$temp,df$icecream)
\end{Sinput}
\begin{Soutput}
	Pearson's product-moment correlation

data:  df$temp and df$icecream
t = 10.4986, df = 10, p-value = 1.016e-06
alternative hypothesis: true correlation is not equal to 0
95 percent confidence interval:
 0.8515370 0.9883148
sample estimates:
      cor 
0.9575066 
\end{Soutput}
\end{Schunk}
%\end{frame}
{\bf Diff btwn cor and cor.test}
The cor.test output also includes the point estimate reported by cor
Cor.test has p-value and also CI

\clearpage

\fontsize{12pt}{14pt}\selectfont

\subsection{Caution}
\begin{frame}{Caution}
\begin{itemize}
\item {\bf !"Correlation Is Not Causation" ... }\newline
When there is a correlation it does not mean that one thing causes the other
\item The magnitude of a correlation depends upon
many factors, including 
\begin{itemize}
\item Sampling (random and representative?)
\item Measurement of X and Y and Several other assumptions 
\ldots
\end{itemize}
\ldots
\end{itemize}
\end{frame}

\section{Assumptions}
\subsection{Assumptions}
\begin{frame}{Assumptions}
\begin{itemize}
    \item Normal Distribution for X and Y if not specifying the method - Use method="Spearman" for non-normal data.
    \item Linear relationship between X and Y
    \item {\bf Homoscedasticity} - homogeneity of variance/ uniformity of variance 
    leveneTest() from car package is used to test this
\end{itemize}
\end{frame}

\end{document}
